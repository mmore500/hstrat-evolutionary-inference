\begin{figure}
  \centering
  \begin{subfigure}[b]{\linewidth}
    \centering
    \includegraphics[width=\textwidth, height=0.16\textheight]{img/plain_resolution_3} \caption{%
      33\% resolution}
    \label{fig:plain-perfect-and-reconstruction-phylogenies:resolution_3}
  \end{subfigure}
  \begin{subfigure}[b]{\linewidth}
    \centering
    \includegraphics[width=\textwidth, height=0.16\textheight]{img/plain_resolution_10}
    \caption{%
      10\% resolution}
    \label{fig:plain-perfect-and-reconstruction-phylogenies:resolution_10}
  \end{subfigure}
  \begin{subfigure}[b]{\linewidth}
    \centering
    \includegraphics[width=\textwidth, height=0.16\textheight]{img/plain_resolution_30}
    \caption{%
      3\% resolution}
    \label{fig:plain-perfect-and-reconstruction-phylogenies:resolution_30}
  \end{subfigure}
  \begin{subfigure}[b]{\linewidth}
    \centering
    \includegraphics[width=\textwidth, height=0.16\textheight]{img/plain_resolution_100}
    \caption{%
      1\% resolution}
    \label{fig:plain-perfect-and-reconstruction-phylogenies:resolution_100}
  \end{subfigure}
  % \begin{noindent}
  \begin{subfigure}[b]{\linewidth}
    \centering
    \includegraphics[width=\textwidth, height=0.16\textheight]{img/reference}
    \caption{%
      reference tree}
    \label{fig:plain-perfect-and-reconstruction-phylogenies:reference}
  \end{subfigure}
  % \end{noindent}
  \caption{%
    Comparison of phylogeny reconstructions produced by the agglomerative trie-based algorithm across different hereditary stratigraphy resolutions for the plain evolutionary regime.
    To maintain visual legibility, these trees contain the same sub-sample of 100 leaf nodes out of the 32,768 in the full trees.
    Sub-figures are arranged from top to bottom in increasing order of reconstruction resolution.
    Taxon and branch color coding is consistent across subpanels.
  }
  \label{fig:plain-perfect-and-reconstruction-phylogenies}
\end{figure}
