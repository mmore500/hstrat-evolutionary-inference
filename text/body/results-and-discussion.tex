\section{Results and Discussion}
\label{sec:results}

\subsection{Phylometric Signatures of Evolutionary Dynamics}

\begin{figure*}
  \begin{minipage}{0.48\textwidth}
    \centering
    Generations Ago (approx.)
  \end{minipage}
  \hfill
  \begin{minipage}{0.48\textwidth}
    \centering
    Generations Ago (approx.)
  \end{minipage}
  \begin{minipage}{0.48\textwidth}
    \hspace{0.02\linewidth}
    \rotatebox{30}{\makebox[0.1\linewidth][c]{200,000}}
    \hfill
    \rotatebox{30}{\makebox[0.1\linewidth][c]{50,000}}
    \hfill
    \rotatebox{30}{\makebox[0.1\linewidth][c]{10,000}}
    \hfill
    \rotatebox{30}{\makebox[0.1\linewidth][c]{2,000}}
    \hfill
    \rotatebox{30}{\makebox[0.1\linewidth][c]{30}}
    \rotatebox{90}{\makebox[0.05\linewidth][c]{0}}
  \end{minipage}
  \hfill
  \begin{minipage}{0.48\textwidth}
    \hspace{0.02\linewidth}
    \rotatebox{30}{\makebox[0.1\linewidth][c]{200,000}}
    \hfill
    \rotatebox{30}{\makebox[0.1\linewidth][c]{50,000}}
    \hfill
    \rotatebox{30}{\makebox[0.1\linewidth][c]{10,000}}
    \hfill
    \rotatebox{30}{\makebox[0.1\linewidth][c]{2,000}}
    \hfill
    \rotatebox{30}{\makebox[0.1\linewidth][c]{30}}
    \rotatebox{90}{\makebox[0.05\linewidth][c]{0}}
  \end{minipage}
  \hfill
  \begin{subfigure}[b]{0.48\textwidth}
    % \begin{noindent}
    \includegraphics[height=0.12\textheight,width=\textwidth]{img/perfect-tree-phylogenies-log/epoch=7+resolution=3+treatment=2.pdf}
    % \end{noindent}
    \caption{%
      strong selection}
    % \label{fig:perfect-tree-phylogenies-log:TODO}
  \end{subfigure}
  \hfill
  \begin{subfigure}[b]{0.48\textwidth}
    % \begin{noindent}
    \includegraphics[height=0.12\textheight,width=\textwidth]{img/perfect-tree-phylogenies-log/epoch=7+resolution=3+treatment=14.pdf}
    % \end{noindent}
    \caption{%
      weak selection}
    % \label{fig:perfect-tree-phylogenies-log:TODO}
  \end{subfigure}
  \hfill
  \begin{subfigure}[b]{0.48\textwidth}
    \centering
    % \begin{noindent}
    \includegraphics[height=0.12\textheight,width=\textwidth]{img/perfect-tree-phylogenies-log/epoch=7+resolution=3+treatment=8.pdf}
    % \end{noindent}
    \caption{%
      plain}
    % \label{fig:perfect-tree-phylogenies-log:TODO}
  \end{subfigure}
  \hfill
  \begin{subfigure}[b]{0.48\textwidth}
    % \begin{noindent}
    \includegraphics[height=0.12\textheight,width=\textwidth]{img/perfect-tree-phylogenies-log/epoch=7+resolution=3+treatment=6.pdf}
    % \end{noindent}
    \caption{%
      spatial structure}
    % \label{fig:perfect-tree-phylogenies-log:TODO}
  \end{subfigure}
  \hfill
  \begin{subfigure}[b]{0.48\textwidth}
    \includegraphics[height=0.12\textheight,width=\textwidth]{img/perfect-tree-phylogenies-log/epoch=7+resolution=3+treatment=26.pdf}
    % \end{noindent}
    \caption{%
      weak 4 niche ecology}
    % \label{fig:perfect-tree-phylogenies-log:TODO}
  \end{subfigure}
  \hfill
  \begin{subfigure}[b]{0.48\textwidth}
    % \begin{noindent}
      \includegraphics[height=0.12\textheight,width=\textwidth]{img/perfect-tree-phylogenies-log/epoch=7+resolution=3+treatment=24.pdf}
    \caption{%
      weak 4 niche ecology with spatial structure }
    % \label{fig:perfect-tree-phylogenies-log:TODO}
  \end{subfigure}
  \hfill
  \begin{subfigure}[b]{0.48\textwidth}
    \includegraphics[height=0.12\textheight,width=\textwidth]{img/perfect-tree-phylogenies-log/epoch=7+resolution=3+treatment=10.pdf}
    % \end{noindent}
    \caption{%
      4 niche ecology}
    % \label{fig:perfect-tree-phylogenies-log:TODO}
  \end{subfigure}
  \hfill
  \begin{subfigure}[b]{0.48\textwidth}
    % \begin{noindent}
    \includegraphics[height=0.12\textheight,width=\textwidth]{img/perfect-tree-phylogenies-log/epoch=7+resolution=3+treatment=22.pdf}
    % \end{noindent}
    \caption{%
      4 niche ecology with spatial structure}
    % \label{fig:perfect-tree-phylogenies-log:TODO}
  \end{subfigure}
  \hfill
  \begin{subfigure}[b]{0.48\textwidth}
    % \begin{noindent}
    \includegraphics[height=0.12\textheight,width=\textwidth]{img/perfect-tree-phylogenies-log/epoch=7+resolution=3+treatment=20.pdf}
    % \end{noindent}
    \caption{%
      8 niche ecology}
    % \label{fig:perfect-tree-phylogenies-log:TODO}
  \end{subfigure}
  \hfill
  \begin{subfigure}[b]{0.48\textwidth}
    % \begin{noindent}
    \includegraphics[height=0.12\textheight,width=\textwidth]{img/perfect-tree-phylogenies-log/epoch=7+resolution=3+treatment=18.pdf}
    % \end{noindent}
    \caption{%
      8 niche ecology with spatial structure}
    % \label{fig:perfect-tree-phylogenies-log:TODO}
  \end{subfigure}
  \hfill
  \caption{%
    Sample reference phylogenies across surveyed evolutionary metrics.
    Each phylogeny has 32,768 leaves.
    Note log-scale $x$ axis.
  }
  \label{fig:perfect-tree-phylogenies-log}
\end{figure*}

% \begin{subfigure}[b]{0.48\textwidth}
% % \begin{noindent}
% \includegraphics[height=0.12\textheight,width=\textwidth]{img/perfect-tree-phylogenies-log/epoch=7+resolution=3+treatment=0/a=collapsed-phylogeny+epoch=00007+mut_distn=np.random.standard_normal+num_generations=32768+num_islands=1024+num_niches=1+p_island_migration=0.01+p_niche_invasion=3.0517578125e-08+population_size=3276.../8+replicate=0+tournament_size=4+treatment=0+_generation=262144+_index=0+scale=log+ext=.pdf}
% % \end{noindent}
% \caption{%
% spatial structure strong election}
% % \label{fig:perfect-tree-phylogenies-log:TODO}
% \end{subfigure}

% \begin{subfigure}[b]{0.48\textwidth}
% % \begin{noindent}
% \includegraphics[height=0.12\textheight,width=\textwidth]{img/perfect-tree-phylogenies-log/epoch=7+resolution=3+treatment=16/a=collapsed-phylogeny+epoch=00007+mut_distn=np.random.standard_normal+num_generations=32768+num_islands=1+num_niches=4+p_island_migration=0.01+p_niche_invasion=3.0517578125e-08+population_size=32768+r.../eplicate=0+tournament_size=1+treatment=16+_generation=262144+_index=16+scale=log+ext=.pdf}
% % \end{noindent}
% \caption{%
% weak selection 4 niche ecology}
% % \label{fig:perfect-tree-phylogenies-log:TODO}
% \end{subfigure}

% \begin{subfigure}[b]{0.48\textwidth}
% % \begin{noindent}
% \includegraphics[height=0.12\textheight,width=\textwidth]{img/perfect-tree-phylogenies-log/epoch=7+resolution=3+treatment=12/a=collapsed-phylogeny+epoch=00007+mut_distn=np.random.standard_normal+num_generations=32768+num_islands=1024+num_niches=1+p_island_migration=0.01+p_niche_invasion=3.0517578125e-08+population_size=3276.../8+replicate=0+tournament_size=1+treatment=12+_generation=262144+_index=12+scale=log+ext=.pdf}
% % \end{noindent}
% \caption{%
% spatial structure weak selection}
% % \label{fig:perfect-tree-phylogenies-log:TODO}
% \end{subfigure}

% \begin{subfigure}[b]{0.48\textwidth}
% % \begin{noindent}
% \includegraphics[height=0.12\textheight,width=\textwidth]{img/perfect-tree-phylogenies-log/epoch=7+resolution=3+treatment=4/a=collapsed-phylogeny+epoch=00007+mut_distn=np.random.standard_normal+num_generations=32768+num_islands=1+num_niches=4+p_island_migration=0.01+p_niche_invasion=3.0517578125e-08+population_size=32768+r.../eplicate=0+tournament_size=4+treatment=4+_generation=262144+_index=4+scale=log+ext=.pdf}
% % \end{noindent}
% \caption{%
% 4 niche ecology strong selection}
% % \label{fig:perfect-tree-phylogenies-log:TODO}
% \end{subfigure}
% \hfill

\begin{figure*}
  \centering
  \begin{subfigure}[b]{\textwidth}
    \includegraphics[width=\textwidth]{binder/binder/teeplots/col=phylometric+epoch=7+mut_distn=np.random.standard_normal+viz=boxplot+x=value+y=regime+ext=.pdf}
  \caption{%
    Distribution of tree phylometrics measured with perfect phylogenetic tracking across surveyed evolutionary regimes under the simple model.
    Sample sizes $n=50$.
  }
  \label{fig:perfect-tree-phylometrics-simple-boxplot}
  \end{subfigure}

  \begin{subfigure}[b]{\textwidth}
    \includegraphics[width=\textwidth]{binder/binder/teeplots/epoch=7+mut_distn=np.random.standard_normal+viz=heatmap+x=regime+y=phylometric+ext=.pdf}
    \caption{
    Evolutionary regimes' effect sizes relative to ``plain'' baseline under the simple model with perfect phylogenetic tracking, normalized via Cliff's delta.
    Sample sizes $n=50$.
    }
\label{fig:perfect-tree-phylometrics-simple-heatmap}
  \end{subfigure}%

\begin{subfigure}[b]{\textwidth}
  \includegraphics[width=\textwidth]{binder/binder/avida-individual/teeplots/col=phylometric+epoch=0+mut_distn=default+viz=boxplot+x=value+y=regime+ext=.pdf}
  \caption{
  Distribution of tree phylometrics measured with perfect phylogenetic tracking across surveyed evolutionary regimes under the Avida model.
  Sample sizes $n=30$.
  }
  \label{fig:perfect-tree-phylometrics-avida-boxplot}
\end{subfigure}%

\begin{subfigure}[b]{\textwidth}
  \includegraphics[width=\textwidth]{binder/binder/avida-individual/teeplots/epoch=0+mut_distn=default+viz=heatmap+x=regime+y=phylometric+ext=.pdf}
\caption{%
  Evolutionary regimes' effect sizes relative to ``plain'' baseline under the Avida model with perfect phylogenetic tracking, normalized via Cliff's delta.
    Sample sizes $n=30$.
}
\label{fig:perfect-tree-phylometrics-avida-heatmap}
\end{subfigure}

  \caption{%
  Tree phylometrics across surveyed evolutionary regimes, calculated on perfect-fidelity simulation phylogenetic records.
  Note that nonparametric effect size normalization caps out to 1.0/-1.0 past the point of complete disbributional nonoverlap.
  For heatmap charts, +'s indicate small, medium, and large effect sizes using the Cliff's delta statistic and *'s indicate statistical significance at $\alpha = 0.05$ via Mann-Whitney U test.
  Results from simple model are for standard experimental conditions: gaussian mutation distribution at epoch 7 (generation 262,144).
  See Figure \ref{fig:perfect-tree-phylometrics-sensitivity-analysis} for results under sensitivity analysis conditions.
  }
  \label{fig:perfect-tree-phylometrics}
\end{figure*}

The feasibility of harnessing phylogenetic analysis to identify evolutionary dynamics hinges on the premise that these dynamics induce detectable structure within the phylogenetic record.
Fortunately, as shown in Figure \ref{fig:perfect-tree-phylogenies-log}, dendrograms of phylogenetic histories from the different evolutionary conditions tested do indeed exhibit striking visual differences.

As a first step to characterizing the phylogenetic impact of spatial structure, ecology, and selection pressure, we tested whether surveyed evolutionary conditions exhibited detectable differences in a representative suite of four phylometrics: evolutionary distinctiveness, Colless-like index, mean pairwise distance, and sum pairwise distance.
Figure \ref{fig:perfect-tree-phylometrics} summarizes the distributions of each metric across surveyed conditions.
Statistical tests confirmed that each phylometric exhibited significant variation among surveyed evolutionary conditions for both the simple model and Avida (Kruskal-Wallis tests; all $p < 10^{-40}$; $n=50$ per condition simple model, $n=30$ Avida; Supplementary Table \ref{tab:phylostatistics-comparison-between-regimes-kwallis}).

To quantify the phylometric effects of surveyed evolutionary regimes, we performed nonparametric statistical comparisons against the ``plain'' baseline treatment.
We used a measure of distributional overlap --- Cliff's delta --- to assess effect sizes, binning into ``negligible'', ``small'' (+), ``medium'' (++), and ``large'' (+++) effects based on conventional thresholds \citep{hess2004robust}.
Significance at $\alpha = 0.05$ (*) was assessed through Mann-Whitney tests.
Figure \ref{fig:perfect-tree-phylometrics-simple-heatmap} shows nonparametric significance and effect size test results.

Relative to the plain regime, all evolutionary regimes in the simple model depress the Colless-like index.
Reduction in this statistic indicates that all deviations from baseline conditions increased regularity in generated phylogenies.
This observation runs somewhat counter to prior results on similar tree balance metrics, in which the presence of spatial structure increased imbalance \citep{scottInferringTumorProliferative2020}.
One possible contributing factor is that taxa in our phylogenies were individuals, whereas Scott et. al used genotype-level abstraction (i.e., their trees were gene trees).
%Application of the metric to trees comprised individual-level taxa, instead of species-level taxa as is the case in most traditional phylogenetics work, may account for this result.
This possible effect of taxonomic unit is consistent with our results from the species-level phylogenies in the  Gen3sis system (Figure \ref{fig:perfect-tree-phylometrics-gen3sis}), in which ecological and spatial conditions elevated Colless imbalance.
Avida individual-level phylogenies were more consistent with the simple model than with Gen3sis; Colless index was significantly depressed under ecological regimes, and weakly but insignificantly depressed under spatial structure.
However, other modes of evolution did not meaningfully affect Colless-like index of Avida phylogenies.

Mean evolutionary distinctiveness was significantly higher under weak selection and with spatial structure than in the plan regime.
This metric significantly decreased under strong selection and under ecological regimes, but the numerical magnitudes of these effects were relatively smaller (Figure \ref{fig:perfect-tree-phylometrics-heatmap-parametric}).
We observed similar results in Avida, except no significant effect of strong selection and weak ecology was detected on the phylometric outcome.

Mean pairwise distance was significantly depressed under all regimes except ecology and weak ecology, although again the numerical magnitude of effects on ecological regimes were relatively smaller (Figure \ref{fig:perfect-tree-phylometrics-heatmap-parametric}).
Within the Avida model, weak but insignificant depressing effects were observed under weak selection and spatial structure regimes.
Interestingly, the strongest depressing effects were observed under the weak ecology and strong ecology regimes.

Finally, sum pairwise distance was significantly increased under all regimes compared to baseline, except for the strong selection regime where it was significantly depressed.
Effect size was again strongest under spatial structure and weak selection (Figure \ref{fig:perfect-tree-phylometrics-heatmap-parametric}).
The Avida model gave similar results, except that no significant effect was detected from the weak ecology and strong selection treatments, with a weak but insignificant increase effect detected under strong selection.

Figure \ref{fig:perfect-tree-phylometrics-simple-heatmap} provides a high-level overview of the magnitude and direction of each regime's effects on evolutionary metrics compared to the plain regime within the simple fitness-explicit model, annotated with significance and effect size information.
Notably, strong and weak selection both significantly decrease Colless-like index and mean pairwise distance but have opposite-sign effects on sum distance and mean evolutionary distinctiveness.
Colless-like distance appears to be the least useful metric in distinguishing evolutionary dynamics, decreasing significantly under all non-plain evolutionary conditions.

Ecological dynamics have significant influence on the surveyed phylometrics.
However, the numerical magnitudes of these effects are generally weakcomparedto spatial structure. and selection effects (Figure \ref{fig:perfect-tree-phylometrics-heatmap-parametric})
So, it appears careful accounting for other evolutionary dynamics (i.e., selection pressure and spatial structure) will be essential to accurate detection of ecology through phylogenetic analysis.
Mean pairwise distance may play a role in identifying ecological dynamics, as ecological dynamics --- in contrast to other factors such as spatial structure and changes in selection pressure --- have weaker effects on this phylometrics.

Phylometric outcomes within Avida generally mirror the simple fitness model, although in many cases phylometric effects are weaker or not significant.
Notably, the strong selection treatment resulted in no significant effects on phylometric values.
Weak selection significantly increased mean evolutionary distinvtiveness and sum distance, as with the simple model, but the effect size was small.
However, spatial structure's effect size on these metrics was large and agreed with the simple model.
The ecology and rich ecology treatments agreed in sign with the simple model and generally had large effect size.
However, unlike the simple model, Avida's weak ecology sole phylometric outcome was a strong, significant decrease in mean pairwise distance.
In contrast, the simple model exhibited no effect on this metric under the weak ecology treatment.

We additionally performed a sensitivity analysis for results from the simple fitness-explicit model over an alternate exponential mutation operator and earlier phylogeny sampling timepoints.
We found the effects of evolutionary conditions on phylometrics to be generally consistent with Figure \ref{fig:perfect-tree-phylometrics-simple-heatmap} across surveyed conditions (Supplementary Figures \labelcref{fig:perfect-tree-phylometrics-sensitivity-analysis,fig:perfect-tree-phylometrics-heatmap-sensitivity-analysis}).

\subsection{Phylometric Signatures of Ecological Dynamics in Spatially Structured Populations}

\begin{figure*}
  \centering
\includegraphics[width=\textwidth]{binder/binder/teeplots/epoch=7+mut_distn=np.random.standard_normal+normed=true+viz=heatmap+x=regime+y=phylometric+ext=.pdf}
\caption{%
  \textbf{Normalized phylometric responses.}
  Heatmap of normalized tree phylometrics across surveyed evolutionary regimes, calculated on perfect-fidelity phylogenies from the simple model.
  Note that normalization shows magnitude of phylometric effect beyond the point of distributional nonoverlap, unlike nonparametric normalization which tops out with complete distributional nonoverlap.
}
  \label{fig:perfect-tree-phylometrics-heatmap-parametric}
\end{figure*}


Evolution within very large populations typically entail elements of spatial structure.
Even within \textit{in silico} contexts, populations will almost inevitably integrate spatial structure that reflects practical limitations of distributed computing hardware \citep{ackley2014indefinitely,moreno2021conduit}.
Therefore, understanding the background effects of spatial structure on the phylogenetic signatures of other evolutionary dynamics will be essential to applications of phylogenetic inference in such applications.
For this analysis, we chose to focus on ecological dynamics due to interest in how their relatively weak phylometric signatures would respond to the relatively strong influence of spatial structure (Figure \ref{fig:perfect-tree-phylometrics-heatmap-parametric}).

\begin{figure*}
  \centering
  \includegraphics[width=\textwidth]{binder/binder/teeplots/col=phylometric+epoch=7+mut_distn=np.random.standard_normal+nuisance=spatial-structure+viz=boxplot+x=value+y=regime+ext=.pdf}
  \caption{TODO}
  \label{fig:perfect-tree-phylometrics-with-spatial-nuisance}
\end{figure*}


Figure \ref{fig:perfect-tree-phylometrics-with-spatial-nuisance} summarizes the distribution of surveyed phylometrics under the three surveyed ecological regimes and the control non-ecological regime, all with spatial population structure.
Annotations indicate results of nonparametric significance and effect size tests against the plain spatial structure treatment, described above for experiments with non-spatial baseline.
Statistical tests confirmed that for the simple model each phylometric exhibited significant variation among these evolutionary regimes (Kruskal-Wallis tests; all $p < 1\times10^{-8}$; $n=50$ per condition for simple model, $n=30$ Avida; Supplementary Tables \labelcref{tab:phylostatistics-comparison-between-regimes-spatial-nuisance-kwallis,tab:phylostatistics-comparison-between-regimes-spatial-nuisance-kwallis-avida}).

Like under the spatially unstructured background, all ecology treatments drove significant increases in mean evolutionary distinctiveness and sum distance.
Also consistent with spatially unstructured results, rich ecology drove significant, large-effect depression of both colless index and mean pairwise distance and ecology.
However, the ecology treatment depressed only meanwise pair distance with spatial structure.
Without spatial structure, the ecology treatment depressed Colless-like index instead.
Results under weak ecology differed notably from non-spatial baseline, with all phylometrics seeing significant, large-effect increases.
Spatial-background rich ecology results from Avida agreed with the simple model.
However, significant phylometric effects were not detected from the ecology and weak ecology treatments under spatial structure conditions.

For these experiments, we again performed a sensitivity analysis over an alternate exponential mutation operator and earlier phylogeny sampling timepoints.
We found the effects of evolutionary conditions on phylometrics to be generally consistent across surveyed conditions (Supplementary Figure \ref{fig:perfect-tree-phylometrics-with-spatial-nuisance-sensitivity-analysis} and Supplementary Tables \labelcref{tab:phylostatistics-comparison-between-regimes-spatial-nuisance-kwallis,tab:phylostatistics-comparison-between-resolutions-allpairs-wilcox-spatial-nuisance}).

\subsection{Species-level Phylogenies}

\begin{figure*}
  \begin{subfigure}[b]{0.5\textwidth}
    \includegraphics[width=\textwidth]{binder/binder/gen3sis/teeplots/epoch=0+mut_distn=default+viz=heatmap+x=regime+y=phylometric+ext=.pdf}
    \caption{non-spatial baseline}
    \label{fig:perfect-tree-phylometrics-heatmap-gen3sis}
  \end{subfigure}%
  \begin{subfigure}[b]{0.5\textwidth}
    \includegraphics[width=\textwidth]{binder/binder/gen3sis/teeplots/epoch=0+mut_distn=default+spatial=true+viz=heatmap+x=regime+y=phylometric+ext=.pdf}
    \caption{spatial baseline}
    \label{fig:perfect-tree-phylometrics-spatial-heatmap-gen3sis}
  \end{subfigure}
  \caption{%
    Evolutionary regimes' effect sizes relative to ``plain'' baseline under the Gen3sis model with perfect phylogenetic tracking, normalized via Cliff's delta.
    Sample sizes $n=30$.
    Annotated +'s indicate small, medium, and large effect sizes using the Cliff's delta statistic and *'s indicate statistical significance at $\alpha = 0.05$ via Mann-Whitney U test.
  }
  \label{fig:perfect-tree-phylometrics-gen3sis}
\end{figure*}
\begin{figure*}
  \centering
\includegraphics[width=\textwidth]{binder/binder/avida/teeplots/epoch=0+mut_distn=default+viz=heatmap+x=regime+y=phylometric+ext=.pdf}
\caption{%
  Tree phylometrics across surveyed evolutionary regimes, calculated on perfect-fidelity simulation phylogenetic records.
  Note that nonparametric effect size normalization caps out to 1.0/-1.0 past the point of complete disbributional nonoverlap.
  For heatmap charts, +'s indicate small, medium, and large effect sizes using the Cliff's delta statistic and *'s indicate statistical significance at $\alpha = 0.05$ via Mann-Whitney U test.
}  
  \label{fig:perfect-tree-phylometrics-heatmap-avida-genome}
\end{figure*}

Figure \ref{fig:perfect-tree-phylometrics-gen3sis} shows phylometric effects of spatial structure and of ecology, both with and without a spatial structure background.
Significant, large-effect changes were detected across all four phylometrics in each treatment.
However, with the exception of sum distance, effect signs of treatments were opposite to those for individual-level phylogenies across all phylometrics.
Tip count effects seem likely to play a role in this discrepancy.
Unlike Avida and the simple fitness-explicit model, which held population size constant, species richess grew freely under the Gen3sis model.
It is also possible that manifested phylometric outcomes may be sensitive to granularity level for the taxonomic unit of phylogeny.
As noted in Section \ref{sec:methods} and shown in Figure \ref{fig:perfect-tree-phylometrics-heatmap-avida-genome}, in Avida experiments application of genome-level tracking (as opposed to individual-level tracking) induced notable phylometric changes, and even flipped signs of some treatment effects.

\subsection{Phylometric Bias of Reconstruction Error}
\label{sec:phylometric-bias-reconstruction-error}

Shifting from perfect phylogenetic tracking to approximate phylogentic reconstruction will facilitate efficiency and robust digital evolution simulations at scale, but introduces a complicating factor into phylogenetic analyses: tree reconstruction error.
A clear understanding of the impact of these errors on the computed phylometrics will be necessary for informative future phylogenetic analyses.

To explore this question, we compared phylometrics computed on reconstructed trees to corresponding true reference trees under the simple fitness-explicit model (Wilcoxon tests; $n=50$ per condition; Supplementary Table \ref{tab:phylostatistics-comparison-between-resolutions-allpairs-wilcox}).
To err towards conservatism in detecting phylometric biases, we did not correct for multiple comparisons.
Reconstructions were performed across a range of precisions, ranging from 1\% relative resolution for MRCA estimates (most precise) to 33\% relative resolution for MRCA estimates (least precise).
Precision was manipulated by adjusting the information content of underlying hereditary stratigraphic genome annotations used to perform phylogenetic reconstruction \citep{moreno2022hereditary}.
Note that important differences exist the between nature of reconstruction error under hereditary stratigraphy versus traditional biosequence-based methods, discussed further below.

\begin{figure*}
  \centering
  \includegraphics[width=\textwidth]{binder/binder/teeplots/epoch=7+hue=quality-threshold+mut_distn=np.random.standard_normal+viz=heatmap+x=regime+y=phylometric+ext=.pdf}
  \caption{Heatmap of the insignificance threshold given by the Wilcoxon signed-rank test of the tree reconstruction error across all surveyed regimes.}
  \label{fig:reconstructed-tree-phylometrics-error}
\end{figure*}


For each phylometric, we sought to determine the minimum resolution required to achieve statistical non-detection (i.e., $p > 0.05$) of bias between reconstructions and their corresponding references.
For nearly all cases, 3\% reconstruction resolution was sufficient to achieve statistical indistinguishability between reference and reconstruction.
Mean evolutionary distinctiveness and sum distance were particularly robust to reconstruction error, showing no detectable bias even at only 33\% reconstruction resolution.

Phylometric sensitivity to reconstruction error was broadly consistent across evolutionary regimes.
Figure \ref{fig:reconstructed-tree-phylometrics-error} summarizes these results.

Where detectable, estimation uncertainty bias decreased all surveyed phylometrics' numerical value.
So, when testing for expected increases in phylometric values, the potential for systematic false positives due to reconstruction error can be discounted.
Supplementary Figure \ref{fig:reconstructed-tree-phylometrics} provides a full comparison the distribution of phylometric estimates on reference trees with the distributions of phylometric estimates for reconstructed trees across reconstruction resolutions.

We performed additional analyses with additional spatially structured ecological evolutionary regimes, over an alternate exponential mutation operator, at earlier phylogeny sampling timepoints, and with the Avida model.
These yielded generally similar findings for the relationship between reconstruction error and phylometric bias (Supplementary Figures \labelcref{fig:reconstructed-tree-phylometrics-error-sensitivity-analysis,fig:reconstructed-tree-phylometrics-error-spatial-nuisance,fig:reconstructed-tree-phylometrics-with-spatial-nuisance,fig:reconstructed-tree-phylometrics-avida,fig:reconstructed-tree-phylometrics-error-avida}; Supplementary Table \labelcref{tab:phylostatistics-comparison-between-resolutions-allpairs-wilcox,tab:phylostatistics-comparison-between-resolutions-allpairs-wilcox-spatial-nuisance}).

Notably, however, in some instances in sensitivty analysis and with Avida reconstruction bias persisted at even 1\% relative resolution.
Forthcoming work has found that the byte-differentia  hereditary stratigraphy configuration used for this experiment tends to lump closely contemporaneous lineage splitting events into a polytomy rather than a double-branching event --- i.e., indicating uncertainty rather than introducing error.
In contrast, reconstructions on single-bit differentia contain erroneously- sequenced branching rather than artefactual polytomies.
Indeed, exploratory work resolving polytomies.
So, future work should explore whether working with single bit differentia could lessen phylometric bias.

Our last objective was to assess how reconstruction error might affect detection of phylometric signatures induced by treatment conditions.
That is, we sought to perform a sort of ``integration test'' for detection of treatment conditions when working with imperfect reconstructions rather than perfect phylogenies.
To this end, we compared the phylometric outcomes of strong/weak selection, spatial structure, ecology, and weak/rich ecology relative to plain conditions for phylogenies reconstructed at 1\%, 3\%, 10\%, and 33\% resolution levels.
Supplemental Figures \labelcref{fig:reconstructed-tree-phylometrics-progressive-heatmap,fig:reconstructed-tree-phylometrics-progressive-heatmap-avida} show heatmaps with sign, effect size, and significance of phylometric effects across gradations of reconstruction precision for the simple fitness-explicit model and Avida, respectively.
In most cases, 3\% resolution sufficed to fully recover phylometric effects of treatments observed with perfectly-tracked phylogenies.
