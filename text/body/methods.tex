\section{Methods} \label{sec:methods}

\subsection{Model System}

genomes (float ``fitness'' value), mutation, and population size

tournament selection

island structure

niche structure

define evolutionary regimes (table?)

sensitivity analysis variables:
mutation distribution
num generations elapsed
... sensitivity analyses are included in the supplement and had generally similar results (I assume, check this) to the main results presented here

we performed 50 replicates of each evolutionary ``regime''

\subsection{Hereditary Stratigraphic Annotations}

the perfect phylogenies were used to generate annotations and then reconstructions so each replicate of the perfect data has directly corresponding replicates of the reconstruction data

be sure to mention recency-proportional resolution, and that 1-byte strata were used (mention how big the annotations are at the end of the runs)

\subsection{Agglomerative, Trie-based Reconstruction}

\subsection{Phylometrics}
\begin{itemize}
    \item Colless-like Index \citep{mirSoundCollesslikeBalance2018}
    \item Mean Pairwise Distance
    \item Num Ancestors
    \item Mean Evolutionary Distinctiveness \citep{isaacMammalsEDGEConservation2007}
\end{itemize}

\subsection{Software and Data Availability}

Example phylogeny-generation, annotation implementation, and reconstruction tools are available in a Production-ready Python package published on PyPi (cite JOSS) A C++ annotation system that will interoperate fully with the Python package is under active development and will be released imminently.
Very explicitly say that we want people to use this software in their own projects?

materials and scripts for these experiments are available at \url{https://github.com/mmore500/hstrat-evolutionary-inference/}. Cite software used (look through requirements.txt dependencies)

how long did phylogeny generation take? how long did reconstruction take?

Data is available via OSF (give link and cite) \url{https://osf.io/vtxwd/}