\section{Conclusion} \label{sec:conclusion}

recap motivation
\begin{itemize}
    \item phylogenetic analysis has great potential to shed light on evolutionary dynamics
    \item however, understanding of the relationship between selection pressure and ecology and phylogenetic structure is nascent
    \item   and the potentialy confounding impact of spatial structure and reconstruction error is not well understood
\end{itemize}

recap work and findings
In this paper, we focus on methodological and theoretical foundations that will be needed to use phylogenetic analyses to observe evolutionary dynamics in complex, distributed artificial life systems.
\begin{enumerate}
    \item need to be able to do fast, accurate tree reconstructions for very large distributed populations
    \item need to understand the relationship between evolutionary dynamics and phylogenetic metrics 
    \item distributed computation introduces a spatial structure nuissance factor, need to understand effect of that on phylogenetic structure
    \item reconstruction introduces estimation error, need to understand effects of that on phylogenetic structure
    \begin{enumerate}
        \item how much precision is required for accurate phylogenetic metrics?
        \item if bias is introduced, what type of bias (so could still detect "conservatively")
    \end{enumerate}
    \end{enumerate}

future work:
\begin{itemize}
    \item better theory and methodology for null models; in this work we performed comparisons between replicates with different system parameterizations; in many contexts, this might not be practical or possible
    \item  incorporate information about ancestral state through ``fossils'' and see how well/accurately they can be tied into the tree of extant organisms (initial signs are good; unit tests with reconstructions from organisms sampled from different phylogenetic depths)
    \item  understanding complications of sexual recombinaton
    \item leverage these foundational observational findings to devise testing methods to identify evolutionary dynamics
\end{itemize}

closing point: we want to develop tools that are applicable across alife systems; software is available and is ready to be used in production; we hope that this can play a role in further developing large-scale (parallel and distributed) artificial life systems as an experimental tool

serialization and deserialization
C++ annotation system that will interoperate fully with the Python package is under active development and will be released imminently.
Very explicitly say that we want people to use this software in their own projects?
Python Packaging Index (PyPi).