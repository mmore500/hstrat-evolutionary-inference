TODO \citep{nozoe2017inferring} incorporate trait data
TODO \citep{volz2013viral} time-sampled phylogenies
TODO normalize for subsampling, population size, and time
TODO graphML / large suites of phylodynamics (note: in alife, we're well-positioned to generate large training sets)

\section{Conclusion}
\label{sec:conclusion}

Traditional phylogenetic analysis revolves around accurately and precisely resolving the sequences of evolutionary events that comprise natural history.
Ongoing work within the field seeks to augment this oeuvre by developing methods to extract information about evolutionary dynamics such as ecology, selection pressure, and spatial structure from these inferred records.
This work contributes to the project of expanding the scope of traditional phylogenetic analysis, seeking in particular to develop the methodological and theoretical foundations that will be needed to apply phylogenetic analyses to observe evolutionary dynamics in complex, distributed artificial life systems.

First, we characterized the effects of selection pressure, ecology, and spatial population structure on phylogeny structure.
Each evolutionary dynamic expressed a distinct signature across phylometrics.
However, compared to spatial structure and selection pressure, ecology exerted relatively muted effects on phylogenetic metrics.
Ecology had no detectable effect on ancestor count and mean pairwise distance.
Ecological influence on mean evolutionary distinctiveness was significant, but small in magnitude compared to effects from the introduction of spatial structure and increased selection pressure.

Surprisingly, follow-up experiments revealed that background spatial population structure can accentuate the phylometric signature of ecology.
Under these conditions, ecology induced sharp increases in ancestor count.
However, the impact of ecology on Colless-like index attenuated with the addition of spatial structure.
These results highlight the complexity of how interacting evolutionary dynamics impact phylogenetic structure.
Even a single dynamic in isolation can influence phylometrics in opposite directions.
For instance, we found that both increasing and decreasing selection pressure can significantly reduce trees' Colless-like index and mean pairwise distance.

Comparing phylometrics of reference trees against corresponding reconstructions, we found that most phylometric statistics were somewhat sensitive to reconstruction error.
Ancestor count was the most sensitive, with reconstructions exhibiting significant bias compared to corresponding references at even 1\% reconstruction resolution.
On the other hand, mean evolutionary distinctiveness was particularly robust to reconstruction error.
No bias was detected for calculations of this metric on reconstructed trees, even when reconstructed with only 33\% resolution.
Most phylometrics reached statistical indistinguishability between reference and reconstruction at or above 3\% reconstruction resolution, suggesting a reasonable ballpark parameterization for applications of hereditary stratigraphy involving phylogenetic analysis.
Additionally, phylometric bias of reconstruction error was generally consistent across different evolutionary regimes, which offers a promising simplification of future experimental considerations.

These findings contribute foundations for development of more rigorous phylogenetic assays that might eventually be capable of identifying the evolutionary conditions that produced a phylogeny.
In particular, the potentially confounding impact of spatial structure and reconstruction error, which before had been poorly characterized, will be relevant to distributed digital evolution systems at scale.
It is especially promising that reconstruction error can be reduced to effectively zero.

Finally, this work introduced, validated, and demonstrated a new phylogenetic reconstruction technique for large-scale populations annotated using hereditary stratigraphy.
This new capability clears an important hurdle to applying hereditary stratigraphy in practice.
Ultimately, the findings presented here represented a multi-pronged attack on the problem of drawing scalable evolutionary inferences from artificial life software.

% Much future work

% \begin{itemize}
% \item better theory and methodology for null models; in this work we performed comparisons between replicates with different system parameterizations; in many contexts, this might not be practical or possible
% \item leverage these foundational observational findings to devise systematic testing protocols to identify evolutionary dynamics
% % \item incorporate information about ancestral state through ``fossils'' and see how well/accurately they can be tied into the tree of extant organisms (initial signs are good; unit tests with reconstructions from organisms sampled from different phylogenetic depths)
% % \item understanding complications of sexual recombinaton

% \end{itemize}

% Ultimately, the goal is to develop methodology that can be put into practice to increase the power of digital evolution and artificial life systems as experimental model systems.
% In addition to , this objective hinges on software availability.

% Production-ready software implementation available via the Python Packaging Index.
% C++ bindings for hereditary stratigraph genome annotation are under active development and will be released imminently.
% We are eager to collaborate to , and to contribute to developing large-scale (parallel and distributed) artificial life systems as an experimental tool

% recap work and findings
% In this paper, we focus on
% \begin{enumerate}
% \item need to be able to do fast, accurate tree reconstructions for very large distributed populations can efficiently create high-quality phylogenetic reconstructions for population sizes as large as 32,768
% \item need to understand the relationship between evolutionary dynamics and phylogenetic metrics
% \item
% \item nuisance factors of spatial structure and phylogenetic reconstruction error accompanying large-scale parallel and distributed digital evolution systems.

% \begin{enumerate}
% \item how much precision is required for accurate phylogenetic metrics?
% \item if bias is introduced, what type of bias (so could still detect "conservatively")
% \end{enumerate}
% \end{enumerate}
