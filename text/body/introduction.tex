\section{Introduction}
\label{sec:introduction}

Within evolutionary biology, the stucture of ancestry relationships among organisms within an evolving system is described as their ``phylogeny.''
Phylogenies detail the sequence of historical lineage-branching events that gave rise to contemporary populations.
Most obviously, the phylogenetic record can reveal the ordering and timing of particular contingencies (e.g., trait originations, population separations) that gave rise to current populations \citep{pagel1999inferring,arbogast2002estimating}.
For instance, phylogenies reconstructed from fungal gene sequences suggest frequent switching of nutritional sources and that spore-dispersal mechanisms have multiple independent origins \citep{james2006reconstructing}.

However, phylogenetic analyses can also test more general hypotheses about underlying evolutionary dynamics.
Rather than seeking to detect specific events, researchers using phylogenies for this purpose instead measure structural patterns that may be indicative of different evolutionary regimes.
For instance, analysis of the count of coexisting lineages over time has been used to detect density-dependent changes in speciation rate \citep{rabosky2008density}, which can be interpreted as an indication of transition away from niche expansion due to lessening availability of ecological opportunity.

Indeed, methods to infer underlying evolutionary dynamics from phylogenetic structure are increasingly harnessed for tangible, impactful applications.
In evolutionary oncology, for example, phylogenetic analysis has been used to quantify patterns of tumor evolution \citep{scottInferringTumorProliferative2020,lewinsohnStatedependentEvolutionaryModels2023}.
Likewise, pathogen phylogenies have been studied to identify dynamics of epidemiological spread of Ebola and HIV \citep{giardina2017inference,saulnier2017inferring,voznica2022deep}.
Phylogenetic information also has a long history of helping identify policy priorities for ecological conservation \citep{forestPreservingEvolutionaryPotential2007}.


In computational evolutionary systems, phylogenetic structure metrics are a powerful tool for summarizing evolutionary history \citep{dolson2020interpreting}.
These analyses can help digital systems further our knowledge of biology, but they are also directly valuable for learning about digital systems themselves.
Notably, within the realm of artificial life research, phylogeny-based metrics have been proposed to identify hallmarks of open-ended evolution \citep{dolsonMODESToolboxMeasurements2019}.
They have also been used to understand evolutionary trajectories that lead to outcomes of interest \citep{lenskiEvolutionaryOriginComplex2003,lalejiniEvolutionaryOriginsPhenotypic2016,johnsonEndosymbiosisBustInfluence2022a}.
For evolution-inspired optimization algorithms, phylogeny-based metrics have been used to predict which runs of evolutionary computation will be successful \citep{hernandezWhatCanPhylogenetic2022a,shahbandeganUntanglingPhylogeneticDiversity2022a} and even proactively guide artificial selection algorithms \citep{lalejini2024phylogeny,burke2003increased}.
% @ELD: Maybe cite Austin's ALife paper here too, once it's citable

Here, we assess the potential of phylogenetic analyis as a means to study fundamental, universal evolutionary dynamics --- ecology, selection pressure, and spatial structure.
Specifically, we consider the following questions,
\begin{enumerate}
  \item Do these dynamics leave a detectable signatures in phylogenetic structure?
  \item If so, to what extent are these dynamics' signatures discernable from one another?
  \item To what extent do the structure of these dynamics' signatures generalize across evolutionary systems?
\end{enumerate}

We use \textit{in silico} experiments to investigate these questions, and test across two levels of taxonomic abstraction.
To quantify phylogenetic structure at organism-level abstraction --- meaning that phylogeny is represented at the granularity of individual parent-child relationships --- we use both a minimalistic, forward-time simulation and a rich artificial life system \citep{ofria2004avida}.
In contrast, at species-level abstraction (i.e. species trees), we used a population-level simulation designed to study origins of biodiversity \citep{hagen2021gen3sis}.

A major strength of simulation experiments for this purpose is that they provide exact ground-truth phylogenies due to their perfectly observable evolutionary history.
However, for \textit{in vivo} systems perfect phylogeny tracking is not typically possible. Although new techniques are making increasingly high-fidelity lineage tracking possible in some experiments \citep{nozoe2017inferring,woodworth2017building}, phylogenies are still usually estimated from naturally-ocurring biosequence data.
Moreover, as digital evolution systems scale, issues of data loss and centralization overhead make perfect tracking at best inefficient and at worst untenable.
Thus, some systems will likely need to adopt a decentralized, reconstruction-based approach similar to biological data \citep{moreno2024analysis}, which can be achieved through the recently-developed ``hereditary stratigraphy'' methodology \citep{moreno2022hstrat}.

Effective use of phylogenetic metrics in scenarios where exact phylogenies are not available requires understanding of potential confounding effects from inaccuracies introduced by reconstruction.
However, it is in precisely such scenarios where the new capabilities to characterize evolutionary dynamics could have the largest impact; large-scale systems can produce an intractable quantity of data, making phylometrics valuable as summary statistics of the evolutionary process \citep{dolsonInferring2020}.
In particular, methods are needed to handle large-scale artificial life systems where complete, perfect visibiltiy is not feasable and evolution operates according to implicit, contextually-dependent fitness dynamics \citep{moreno2022exploring,kojima2023implementation}.
Thus, an additional goal of this paper is to quantify the magnitude and character of bias that reconstruction error introduces.
In particular, we calculate phylometric metrics across range of reconstruction accuracy levels, and report the level of accuracy necessary to attain metric readings statistically indistinguishable from ground truth.

The utility of a model system for experimental evolution hinges on sufficient ability to observe and interpret underlying evolutionary dynamics.
Benchtop and field-based biological models continue to benefit from ongoing methodological advances that have profoundly increased visibility into genetic, phenotypic, and phylogenetic state \citep{woodworth2017building,blomberg2011measuring,schneider2019past}.
Simulation systems, in contrast, have traditionally enjoyed perfect, complete observability \citep{hindre2012new}.
However, ongoing advances in parallel and distributed computing hardware have begun to strain analytical omnipotence, as increasingly vast throughput introduces challenges centralizing, storing, and analyzing data \citep{klasky2021data}.
These trends, although unfolding opposite one another, exhibit intruiging convergence: computational and biological domains will provide data that is multimodal and high-resolution but also incomplete and imperfect.
Such convergence establishes great potential for productive interdisciplinary exchange.

% These come in two forms: 1) that can be used to draw abstract insights from lineages and phylogenies \citep{dolsonInterpretingTapeLife2020}, and 2) algorithmic techniques for efficiently collecting phylogenetic data at scale \citep{morenoHereditaryStratigraphyGenome2022}.

% Currently, most research on this topic is either very abstract (e.g., the presence of negative-frequency dependent selection increases phylogenetic diversity) or very specific (e.g., phylogenetic structure can be used to infer cell division rates in solid tumors \citep{lewinsohnStatedependentEvolutionaryModels2023}).
% Here, we lay the groundwork for identifying the fingerprints left on phylogenetic structure by a larger range of evolutionary dynamics in a scalable and robust manner.

%Some groundwork has been done in the fields of conservation biology and cancer biology, as well as in artificial life (cite ``tape of life''?) to understand how evolutionary dynamics affect phylogenetic structure (and the fingerprints of evolutionary dynamics on phylogenetic structure) but this is still a rapidly emerging field (be sure to hedge here).
%In particular, existing work has shown x, y and z.

% %One major difference between digital study systems and biological study systems is the availability of perfect data.
% %In biology, we rely on reconstructions to estimate phylogenetic history; the effect of estimation error on phylometrics is poorly understood.
% Historically, digital evolution systems have had the advantage of providing perfectly accurate phylogenetic data, in contrast to the reconstructed phylogenies relied upon in traditional biology.
% However, as digital evolution systems scale, issues of data loss and decentralization will make perfect tracking at best inefficient and at worst untenable.
% Thus, some systems will likely need to adopt a decentralized, reconstruction-based approach similar to biological data.
% %Decentralization will also likely introduce an aspect of spatial structure (i.e., populations will no longer be ``well-mixed'') which will further complicate phylogenetic analyses.
% Consequently, if our goal is to use phylogenetic metrics to increase the scalability of our data analysis, we must also understand how robust they are to inaccuracies introduced by reconstruction.
% % (maybe briefly cite/mention effect of spatial structure on phylogenies)

% In this paper, we build the following methodological and theoretical foundations that will be needed to use phylogenetic analyses to observe evolutionary dynamics in complex, decentralized systems with imperfect observability:
% \begin{enumerate}
%   \item understanding of the relationship between evolutionary dynamics and phylogenetic metrics,
%   \item quantification of the effects of reconstruction-induced estimation error on phylogenetic structure, both in terms of the amount of precision required for accuracy and the amount of bias introduced by inadequate precision, and
%   \item quantification of the phylogenetic effects of spatial structure, as distributed computation does not support well-mixed populations.
% \end{enumerate}

% \begin{enumerate}
% \item how much precision is required for accurate phylogenetic metrics?
% \item if bias is introduced, what type of bias (so could still detect "conservatively")
% \end{enumerate}

% In evolutionary biology, phylogenetic history is a valuable tool to triangulating the context of notable evolutionary events, as well as characterizing the underlying mode and tempo of evolution.
% The same holds true in evolutionary simulation.
% % https://github.com/mmore500/phylotrack-algorithm-analysis/blob/fe16d2b2d7df99faade09c01b72f681160749f51/tex/text/body/introduction.tex
% In addition to addressing questions of natural history, access to the phylogenetic record of biological life has proven informative to conservation biology, epidemiology, medicine, and biochemistry among other domains \citep{faithConservationEvaluationPhylogenetic1992, STAMATAKIS2005phylogenetics, frenchHostPhylogenyShapes2023,kim2006discovery}.
% Nonetheless, existing analyses of phylogenetic structure within digital systems have already proven valuable, enabling diagnosis of underlying evolutionary dynamics \citep{moreno2023toward,hernandez2022can,shahbandegan2022untangling, lewinsohnStatedependentEvolutionaryModels2023a} and even serving as mechanism to guide evolution in application-oriented domains \citep{lalejini2024phylogeny,lalejini2024runtime,murphy2008simple,burke2003increased}.
% Further, comparison of observed phylogenies against simulation phylogenies can be used to evaluate hypotheses for underlying dynamics within real-life evolutionary/epidemiological systems \citep{giardina2017inference,voznica2022deep}.
