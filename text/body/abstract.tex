% Here, we assess the potential of phylogenetic analyis as a means to study fundamental, universal evolutionary dynamics --- ecology, selection pressure, and spatial structure.
% Specifically, we consider the following questions,
% \begin{enumerate}
%   \item Do these dynamics leave a detectable signatures in phylogenetic structure?
%   \item If so, to what extent are these dynamics' signatures discernable from one another?
%   \item To what extent do the structure of these dynamics' signatures generalize across evolutionary systems?
% \end{enumerate}

% We use simulation experiments to investigate these questions.
% To quantify organism-level phylogenetic structure within asexual populations, we measure phylogenetic summary statistics from both a minimalistic, forward-time simulation and a rich artificial life system \citep{ofria2004avida}.
% We additionaly contrast organism-level simulation with species phylogenies generated from a population-level simulation \citep{hagen2021gen3sis}.



 % for observing
 % Ultimately, these contributions advance phylogenetic observation as a promising technique for key evolutionary drivers in large-scale populations within closer reach.

%As digital evolution systems grow in scale and complexity, observing and interpreting their evolutionary dynamics will become increasingly challenging.
%Distributed and parallel computing, in particular, introduce obstacles to maintaining the high level of observability that makes digital evolution a powerful experimental tool.
%Phylogenetic analyses represent a promising tool for drawing inferences from digital evolution experiments at scale.
%Recent work has introduced promising techniques for decentralized phylogenetic inference in parallel and distributed digital evolution systems.
%However, foundational phylogenetic theory necessary to apply these techniques to characterize evolutionary dynamics is lacking.
%Here, we lay the groundwork for practical applications of distributed phylogenetic tracking in three ways: 1) we present an improved technique for reconstructing phylogenies from tunably-precise genome annotations, 2) we begin the process of identifying how the signatures of various evolutionary dynamics manifest in phylogenetic metrics, and 3) we quantify the impact of reconstruction-induced imprecision on phylogenetic metrics.

\begin{abstract}
  Evolutionary dynamics are shaped by a variety of fundamental, generic drivers, including spatial structure, ecology, and selection pressure.
  These drivers impact the trajectory of evolution, and have been hypothesized to influence phylogenetic structure.
  For instance, they can help explain natural history, steer behavior of contemporary evolving populations, and influence efficacy of application-oriented evolutionary optimization.
  Likewise, in inquiry-oriented artificial life systems, these drivers constitute key building blocks for open-ended evolution.
  Here, we set out to assess (1) if spatial structure, ecology, and selection pressure leave detectable signatures in phylogenetic structure, (2) the extent, in particular, to which ecology can be detected and discerned in the presence of spatial structure, and (3) the extent to which these phylogenetic signatures generalize across evolutionary systems.
  To this end, we analyze phylogenies generated by manipulating spatial structure, ecology, and selection pressure within three computational models of varied scope and sophistication.
  We find that selection pressure, spatial structure, and ecology have characteristic effects on phylogenetic metrics, although these effects are complex and not always intuitive.
  Signatures have some consistency across systems when using equivalent taxonomic unit definitions (e.g., individual, genotype, species).
  Further, we find that sufficiently strong ecology can be detected in the presence of spatial structure.
  We also find that, while low-resolution phylogenetic reconstructions can bias some phylogenetic metrics, high-resolution reconstructions recapitulate them faithfully.
  Although our results suggest potential for evolutionary inference of spatial structure, ecology, and selection pressure through phylogenetic analysis, further methods development is needed to distinguish these drivers' phylometric signatures from each other and to appropriately normalize phylogenetic metrics.
   With such work, phylogenetic analysis could provide a versatile toolkit to study large-scale evolving populations.
\end{abstract}
