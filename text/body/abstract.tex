\begin{abstract}
  This extended abstract summarizes our upcoming work, ``Toward Phylogenetic Inference of Evolutionary Dynamics at Scale.''
  As evolutionary computing systems grow in scale and complexity, observing and interpreting their evolutionary dynamics will become increasingly challenging.
  Distributed and parallel computing, in particular, introduce obstacles to maintaining the high level of observability necessary to understand and tune evolutionary computing algorithms.
  Phylogenetic analyses represent a promising tool to meet these challenges.
  Recent work has introduced promising techniques for decentralized phylogenetic inference in parallel and distributed digital evolution systems.
  However, foundational phylogenetic theory necessary to apply these techniques to characterize evolutionary dynamics is lacking.
  Reported results lay the groundwork for practical applications of distributed phylogenetic tracking: we describe signatures of various evolutionary dynamics on phylogenetic metrics and quantify the impact of reconstruction-induced error on phylogenetic metrics.
  We find that selection pressure, spatial structure, and ecology have distinct effects on phylogenetic metrics, although these effects are complex and not always intuitive.
  We also find that, while low-resolution phylogenetic reconstructions can bias some phylogenetic metrics, high-resolution reconstructions generally recapitulate them faithfully.
\end{abstract}
