\begin{abstract}
Phylogenetic analyses represent a promising tool for drawing inferences from digital evolution experiments at scale. Here, we expand on this promise in multiple ways: 1) we begin the process of identifying how the signatures of various evolutionary dynamics manifest in phylogenetic metrics, 2) we present an improved technique for reconstructing phylogenies from tunably-precise genome annotations, and 3) we quantify the impact of reconstruction-induced imprecision on phylogenetic metrics. Ultimately, we find that different evolutionary regimes indeed have distinct effects on phylogenetic metrics, although these effects are complex and not always intuitive. We also find that, while low resolution phylogenetic reconstructions can bias some phylogenetic metrics, high resolution reconstructions recapitulate them faithfully. 
\end{abstract}
