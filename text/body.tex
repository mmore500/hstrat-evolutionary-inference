\vspace{-1.5ex}
\section{Introduction}

Phylogenies (i.e., ancestry trees) reveal how evolving populations move through a search space, which offers a powerful window into understanding the behavior of evolutionary algorithms.
In existing work, phylogenetic analyses have diagnosed conditions stymying adaptive evolution, predicted future adaptive success of evolutionary runs, and revealed the step-by-step process by which an evolutionary algorithm solves a problem \citep{hernandezWhatCanPhylogenetic2022a,shahbandeganUntanglingPhylogeneticDiversity2022a,lalejiniEvolutionaryOriginsPhenotypic2016}.
As evolutionary computation (EC) systems grow in complexity and computational scale  (i.e., parallel and distributed computing), observability enabled through phylogenetic analyses will become increasingly crucial.

Existing phylogenetic methods in EC assume a perfect-tracking model where an exact parentage record enables direct inspection of the lineages without uncertainty or inaccuracy.
Applications to large-scale systems, however, face practical difficulties: communication overhead from record centralization and fragility to data loss.
In contrast, work in biological phylogenetics \textit{estimates} history using extant information from a fully-distributed system.
The recently-introduced ``hereditary stratigraphy'' method facilitates analogous reconstruction-based phylogenetic analysis of distributed EC systems.
This technique works by attaching heritable annotations to individual digital genomes such that trade-off between annotation size and estimation accuracy can be directly tuned \citep{moreno2022hereditary}.

Here, we report follow-up work developing methodological foundations necessary to apply hereditary stratigraphy to observe evolutionary dynamics in complex, distributed digital evolution systems.
First, we characterize the impact of ecology, selection pressure, and spatial structure on phylogenetic metrics.
Second, we quantify the effects of reconstruction-induced estimation error on these phylogenetic metrics.
Third, because distributed methods generally do not support well-mixed populations, we investigate the phylometric signature of interactions of ecology and selection pressure with spatial structure.

\vspace{-1.5ex}
\section{Methods}

Experiments testing the relationships between evolutionary dynamics, reconstruction error, and phylogenetic structure required a model system amenable to direct, interpretable tuning of ecology, spatial structure, and selection pressure.
A parsimonious model system was devised to fulfill these objectives.
Genomes in this system comprised a single floating-point value, with higher magnitude corresponding to higher fitness.
Experiments used the following parameterizations: Gaussian mutation, population size 32,768 ($2^{15}$), 262,144 ($2^{18}$) synchronous generations, and tournament selection.

Treatments explored three evolutionary variables: selection pressure (via tournament size), spatial structure (via an island model with 1D closed-ring topology and 1\% migration rate), ecology (via a simple niche model with niche swap probability $3 \times 10^{-8}$ so one niche swap would be expected every 1,000 generations).
We combined these variables into seven ``regimes'' of evolutionary conditions:

\begin{minipage}[t]{\columnwidth}
\begin{itemize}
  \item \textit{plain}: tournament size 2 with no niching and no islands,
  \item \textit{weak selection}: tournament size 1 with no niching and no islands,
  \item \textit{strong selection}: tournament size 4 with no niching and no islands,
  \item \textit{spatial structure}: tournament size 2 with no niching and 1,024 islands,
  \item \textit{weak 4 niche ecology}: tournament size 2 with 4 niches and niche swap probability increased 100$\times$,
  \item \textit{4 niche ecology}: tournament size 2 with 4 niches, and
  \item \textit{8 niche ecology}: tournament size 8 with 4 niches.
\end{itemize}
\end{minipage}

Sensitivity analysis to evolutionary duration (i.e., number generations) and mutation operator (via an alternate unit-exponential mutation distribution) support our findings' generalizability.

Our phylogenetic analyses employ four metrics: (1) \textit{internal node count}, a measure of phylogenetic richness impacted by ecology and spatial structure; (2) \textit{Colless-like index}, an indicator of tree imbalance associated with varying ecological pressures and spatial structure; (3) \textit{mean pairwise distance}, a metric of evolutionary divergence affected by factors promoting long-term maintenance of distinct branches and factors promoting diversity; and (4) \textit{evolutionary distinctiveness}, another metric of evolutionary divergence calculated at the level of individual taxa --- known to behave differently than mean pairwise distance with a particularly strong relationship to branch length \citep{tuckerGuidePhylogeneticMetrics2017}.

Experiments investigating the impact of phylogenetic inference error on phylometrics test four trade-off levels between resolution and genome annotation size.
Each level is described as a $p\%$ ``resolution'' meaning that uncertainty (generational distance between reference points) $k$ generations back is less than $(p / 100) \times k$. So, a high percentage $p$ indicates coarse resolution and a low percentage $p$ indicates fine resolution.
Annotation size ranged from 68 1-byte fingerprints per genome at coarse 33\% resolution to 1,239 1-byte fingerprints at fine 1\% resolution.

Supporting materials are available at \url{https://osf.io/vtxwd/}.
This project benefited from open-source scientific software \citep{ofria2020empirical,moreno2022hstrat,lalejini2019data,sukumaran2010dendropy}.

\vspace{-1.5ex}
\subsection{Summary of Results}

Each of the four surveyed phylometrics exhibited significant variation between surveyed evolutionary conditions.
Ecological dynamics had a significant but relatively weak influence on phylometrics, suggesting the need to carefully account for selection pressure and spatial structure to ensure accurate detection of ecology through phylogenetic analysis.
The Colless-like index appeared to be less useful for distinguishing evolutionary dynamics: it decreased under all non-plain evolutionary conditions.
Figure \ref{fig:perfect-tree-phylometrics} summarizes the distributions of each phylometric across surveyed evolutionary conditions.

Follow-up experiments tested whether ecological dynamics could still be detected in the presence of spatial structure.
We found this to be the case.
Interestingly, spatial structure appeared to mediate some aspects of ecological phylogenetic structure, which did not appear in its absence.
For instance, ecology had a much larger impact on ancestor count in the presence of spatial structure.
However, spatial structure muted the effects of ecology on the Colless-like index.
These findings highlight the background effects of spatial structure as key to interpretation of phylogenetic signatures of other evolutionary dynamics.

Finally, we analyzed the impact of phylogenetic reconstruction error on phylogenetic analysis.
At the coarsest reconstruction resolution, we detected a significant effect of reconstruction error on all phylometrics except mean evolutionary distinctivenes.
Colless-like index and mean pairwise distance generally reached statistical indistinguishability (at $n=50$) at 3\% reconstruction resolution.
Ancestor count, however, was highly sensitive to reconstruction error.
Future work should investigate the possibility of reducing this sensitivity by resolving polytomies (overrepresented due to reconstruction uncertainty) into sets of bifurcations.
Where detectable, estimation uncertainty bias decreased all surveyed phylometrics' numerical value.
So, when testing for expected increases in phylometric values, the potential for systematic false positives due to reconstruction error can be discounted.

\vspace{-1.5ex}
\section{Conclusion}

Phylogenetic analysis appears feasible as means to anatomize complex, distributed evolutionary computing systems.
This work characterizes the phylometric signatures of selection pressure, ecology, spatial population structure, and reconstruction error.
These findings offer practical, actionable basis for phylogenetic inference of evolutionary dynamics and serves as for starting point for development of a systematic methodology for such analyses.
